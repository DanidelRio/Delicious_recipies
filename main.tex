% \documentclass[12 pt]{article}
% \usepackage[utf8]{inputenc}
% \usepackage[margin=1in]{geometry}
% \renewcommand{\baselinestretch}{1}
% \usepackage{fontspec}
% \setmainfont{Times New Roman}
\documentclass[12pt]{article}
\usepackage[utf8]{inputenc}
\usepackage{graphicx}
\usepackage{amssymb}
\usepackage{amsmath}
\usepackage{subcaption}
\usepackage{setspace}
\usepackage[spanish, es-tabla]{babel}
\usepackage{csquotes}
\usepackage{pdfpages} % incluir archivos en pdf
\usepackage{hyperref} % hyperrefs setup
\hypersetup{
  colorlinks=true,
  urlcolor=blue,
  linkcolor=blue,
  pdftoolbar=true,
  pdfmenubar=true,
  citecolor=blue,
}

\usepackage[margin=2.5cm]{geometry}
\usepackage{fancyhdr} %for headers and footers
\pagestyle{fancy}


\title{Pozole}
\author{ddelrio }
\date{October 2020}

% \setlength\itemsep{-0.5em}

\begin{document}
Pozole

Youtube videos:
\begin{itemize}
\setlength\itemsep{0em}
    \item \url{https://www.youtube.com/watch?v=00PQth78T2A}%{a ver que pasa}
    \item \url{https://www.youtube.com/c/JaujaCocinaMexicana/featured}
\end{itemize}

Ingredients:
\begin{itemize}
\setlength\itemsep{0em}
    \item 4 big chiles guajillos (abue also recommends marasol) around 1/4 Kg (large chilis aren't as hot/spicy compared to smaller ones)
    \item 1 chile ancho
    \item 4 laurel leaves
    \item Oregano
    \item Onion
    \item Garlic
    \item Lettuce
    \item Radishes
    \item Lime
    \item Tostadas
    \item Avocado
    \item Precooked Corn grains (1 kg?) (no yellow head)
    \item Chicken or pork meat (if there are bones or fat, this gives taste)
    \item Cream
    \item Chicharron?
    \item Salt
    \item Pepper
    \item Knorr suiza
\end{itemize}

Kitchen utensils:
\begin{itemize}
\setlength\itemsep{0em}
    \item Pan (chili sauce)
    \item 2 or 3 large pots (7 liters?) (one for corn, another for white, the other for red pozole)
    \item Blender
    \item Strainer
    \item Cutting board
    \item Bowls we are going to serve the pozole in
    \item Spoons for all!
\end{itemize}

Note: If you need to add water, ALWAYS hot water!

Steps:
\begin{enumerate}
\setlength\itemsep{0em}
    \item Wash the corn grains around 5 times with cold water until they stop emitting a white substance. (This step is only necessary if the corn is not pre-cooked) 
    \item Put water to boil, and when it is boiling, add corn grains. Cook them with onion and NO salt, in case they are not precooked. Until they start exploding (like popcorn, 1-2 hours) you add salt and let it boil until it is well cooked and with good flavor.
 While the corn is cooking...
 0. Let water boil in a pan.
  1. Put chilis in that pan, when they are soft (around 10-20 min), take out the veins. Or take out the veins before putting in pan.
 2. Put chilis in blender with water, garlic, salt, onion, pepper and knorr suiza, you blend it and grind it (colarlo)?
 3. Put this in a pan with a little bit of oil with small flame for chilis to liberate their taste.
 0. Wash the meat with cold water.
 1. In another pot, put chicken or pork with water, onion and garlic, a little salt and maybe add laurel leaves. When the meat is cooked, take it out and shread it. When In the broth, you put the chile colado and let it boil.  We could divide this in 2 pots, for one to be white and the other red pozole.
 2. When this is boiling, you add the corn and salt to taste, and it is ready.
 1. When you serve, there is where you add the meat.
 1. Cut onion, radishes, lettuce (fine strips), avocado, lime, oregano, cream and tostadas.
\end{enumerate}
  



\end{document}